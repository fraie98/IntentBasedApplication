\chapter{Testing}
\section*{Languages and Frameworks}
In order to test our system, we need to use a tool to create a \textit{virtual network} inside our machine; we used \texttt{mininet}, 
exploiting the \texttt{python2} APIs. To simulate input from an external application we also used a python2 library, called \texttt{requests},
which is able to act like an HTTP client.\
Given that our developement environment is inside a virtual machine, we will keep the number of virtual devices relatively small, in order to not
overload the VM resources.

\section*{Scenario}
We want to simumlate a tipical data center scenario with a single LAN, implemented in a \textit{Spine-Leaf} topology. This configuration
is widely used thanks to it's easy scalability and sufficent redundancy.
\begin{figure}[h]
    \centering
    \caption{example of \textit{spine-leaf} topology}
    \includegraphics[width=0.90\textwidth]{img/spine_leaf.pdf}
\end{figure}

Using \texttt{mininet}, we can also simulate an episode of link failure, in order to test the system behaviour in this specific case. The system is able to
recompute a functional path between two host and use it to forward packets.

\section{Ping Latency}
In order to mesure the network performance we will consider a fixed scenario with 2 spines, 3 leafs and 4 hosts for each leafs (i.e. 12 hosts in total);
for every host:
\begin{itemize}
    \item  we establish an intent with another randome chosed one;
    \item  we start a ping session, measuring the \texttt{round trip time} of each packet
    \item  the session end when 10 ping are succesfully exchanged, or the last timeout elapse
\end{itemize} 
\begin{figure}[ht]
    \centering
    \includegraphics[width=.92\textwidth]{img/mean_ping_time.pdf}
    \caption{ping times with 12 hosts}
\end{figure}
First of all we motitor the number of \texttt{DESTINATION HOST UNREACHABLE} and \texttt{DUPLICATE}; those 2 parameters are at 0, confirming the correct
behaviourof the network.
The performance results are consistent with our expectation:
\begin{itemize}
    \item the first ping is significantly slower than the others, because it's processed by the controller, triggering the route establishment;
    \item subsequent pings are very fast (<1 ms), because they don't have to traverse "real" network cables.
\end{itemize}