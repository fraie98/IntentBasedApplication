% !TeX spellcheck = en_US
\chapter{Testing}
\section*{Languages and Frameworks}
In order to test our system, we need to use a tool to create a \textit{virtual network} inside our machine: we used \texttt{mininet}\footnote{http://mininet.org/}, 
exploiting the \texttt{python2} APIs. To simulate input from an external application we also used a python2 library, called \texttt{requests}\footnote{https://docs.python-requests.org/en/latest/},
which is able to act like an HTTP client.\
Given that our developement environment is inside a virtual machine, we will keep the number of virtual devices relatively small, in order to not
overload the VM resources.

\section*{Scenario}
We want to simulate a typical data center scenario with a single LAN, implemented in a \textit{Spine-Leaf} topology. This configuration
is widely used thanks to its easy scalability and sufficient redundancy.
\begin{figure}[h]
    \centering
    \caption{example of \textit{spine-leaf} topology}
    \includegraphics[width=0.90\textwidth]{img/spine_leaf.pdf}
\end{figure}

Using \texttt{mininet}, we can also simulate an episode of link failure, in order to test the system behaviour in this specific case. The system is able to recompute a functional path between two host and use it to forward packets.
\newpage
\section{Ping Latency}
In order to measure the network performance we will consider a fixed scenario with 2 spines, 3 leafs and 4 hosts for each leafs (i.e. 12 hosts in total);
for every host:
\begin{itemize}
    \item  we establish an intent with another random chosen one;
    \item  we start a ping session, measuring the \texttt{round trip time} of each packet
    \item  the session end when 10 ping are successfully exchanged, or the last timeout elapses
\end{itemize} 
\begin{figure}[h]
    \centering
    \includegraphics[width=.92\textwidth]{img/mean_ping_time.pdf}
    \caption{Ping times with 12 hosts}
    \label{img:perf1}
\end{figure}
First of all we monitor the number of \texttt{DESTINATION HOST UNREACHABLE} and \texttt{DUPLICATE}, those 2 parameters are equal to 0, confirming the correct
behaviour of the network.
The performance results (see figure \ref{img:perf1}) are consistent with our expectation:
\begin{itemize}
    \item the first ping is significantly slower than the others, because it is processed by the controller, triggering the route establishment;
    \item subsequent pings are very fast (<1 ms), because they don't have to traverse "real" network cables and because they are processed by the virtual switches and not by the controller.
\end{itemize}
\newpage
\begin{figure}[h]
    \centering
    \includegraphics[width=.94\textwidth]{img/increasing_ping_time.pdf}
    \caption{Ping times with different loads}
    \label{img:perf2}
\end{figure}
\noindent We also tried to evaluate the elasticity of the network with increasing load of traffic (i.e. increasing number of host pinging at the same time) and the results of this tests can be seen in figure \ref{img:perf2}.\\
The time needed for the first ping to complete tends to increase with the amount of traffic meanwhile the time for the others pings remains pretty stable
(mind the confidence intervals). This is expected because route establishment is done by the controller in a centralized way (the requests will queue up);
once the route is established, switches will handle the traffic in a distributed manner, leading to optimal forwarding of traffic.