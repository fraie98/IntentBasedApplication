% !TeX spellcheck = en_US
\documentclass[a4paper]{report}
\usepackage[T1]{fontenc}
\usepackage[utf8]{inputenc}
\usepackage[english]{babel}
\usepackage{geometry}
\usepackage{graphicx}
\usepackage{subfig}
%\usepackage{lipsum}
\usepackage{verbatim}
\usepackage[table,xcdraw]{xcolor}
\geometry{a4paper,top=2.5cm,bottom=2.5cm,left=3cm,right=3cm,%
	heightrounded,bindingoffset=5mm}

\usepackage{color}
\usepackage{listings}
\usepackage{xcolor}


\newcommand{\HRule}{\rule{\linewidth}{0.5mm}}

\begin{document}
	\begin{titlepage}
		\begin{center}
			
			% Top 
			\includegraphics[width=0.45\textwidth]{img/unipi.png}~\\[2.5cm]
			
			
			% Title
			\HRule \\[0.4cm]
			{ \LARGE 
				\textbf{Intent Based Application}\\[0.4cm]
				\emph{Group Project Report for Advanced Network Architectures And Wireless Systems}\\[0.4cm]
			}
			\HRule \\[1.5cm]
			
			
			
			% Author
			{ \large
				Francesco Iemma \\[0.1cm]
				Yuri Mazzuoli \\[0.1cm]
				Giovanni Menghini \\[0.1cm]
			}
			
			\vfill
			
			\textsc{\large M.Sc. in Computer Engineering}\\[0.4cm]
			
			
			% Bottom
			{\large Academic Year 2021/22}
			
		\end{center}
	\end{titlepage}
	
	
	\tableofcontents
	\newpage
	
	\chapter*{Introduction}
	This project is aimed to design and develop an intent based application. The scenario is the following: \textit{"Consider a set of clients that can communicate through a redundant network
	(e.g., based on a spine-leaf topology.) An external application can request to install paths
	between host pairs, by just specifying the endpoint identifiers, we refer to this as an
	host-2-host intent. The network has to allow communications only among the specified host
	pairs. Moreover, the network has to automatically reconfigure in case of link failures.
	Design and realize a system that allows users to request and withdraw host-to-host intents,
	and configures the underlying network accordingly."}

	
	\noindent The objectives of the projects are:
	\begin{enumerate}
		\item \textit{"Implement a Floodlight module that exposes a RESTful interface allowing clients to create/delete host-to-host instents. The module will then dynamically install and update flow rules in the network to allow the communication among specified pairs. Possible path switches must occur transparently to clients."}
		
		\item \textit{"Test the overall system using Mininet and Floodlight, devising proper scenarios to demonstrate the above functionalities."}
	\end{enumerate}
	
	
	\chapter{Implementation}
	\section{Scenario}
	\noindent The objective of the intent based application is mainly to expose a REST interface to allow the request of an \textbf{intent}. An \textbf{intent} is a request, done by an host, to have a link with another host, this link must tolerate link failures.
	\noindent Hence the controller must implement a module that is in charge of:
	\begin{itemize}
		\item Computing the best path between hostA and hostB
		\item Installing in the switches of the network the proper rules to establish this path
		\item Being responsive in case of a link failures and establishing a new path to maintain the link alive
	\end{itemize}
	\noindent In the implementation of the application we have re-used some Floodlight modules extending some classes where needed. In the following sections we will analyze the choices done and the classes extended in order to implement the features shown previously.
	
	\section{How To Compute A Path: Routing Service}
	
	\section{How To Install A Path: Forwarding Base}
	
	\section{Floodlight Forwarding Module}
	
	\section{Responsiveness To Link Failures \& Topology Changes}
	
	
	\chapter{Testing}
	
\end{document}